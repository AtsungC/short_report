\documentclass[]{elsarticle} %review=doublespace preprint=single 5p=2 column
%%% Begin My package additions %%%%%%%%%%%%%%%%%%%
\usepackage[hyphens]{url}



\usepackage{lineno} % add
\providecommand{\tightlist}{%
  \setlength{\itemsep}{0pt}\setlength{\parskip}{0pt}}

\usepackage{graphicx}
%%%%%%%%%%%%%%%% end my additions to header

\usepackage[T1]{fontenc}
\usepackage{lmodern}
\usepackage{amssymb,amsmath}
\usepackage{ifxetex,ifluatex}
\usepackage{fixltx2e} % provides \textsubscript
% use upquote if available, for straight quotes in verbatim environments
\IfFileExists{upquote.sty}{\usepackage{upquote}}{}
\ifnum 0\ifxetex 1\fi\ifluatex 1\fi=0 % if pdftex
  \usepackage[utf8]{inputenc}
\else % if luatex or xelatex
  \usepackage{fontspec}
  \ifxetex
    \usepackage{xltxtra,xunicode}
  \fi
  \defaultfontfeatures{Mapping=tex-text,Scale=MatchLowercase}
  \newcommand{\euro}{€}
\fi
% use microtype if available
\IfFileExists{microtype.sty}{\usepackage{microtype}}{}
\bibliographystyle{elsarticle-harv}
\usepackage{color}
\usepackage{fancyvrb}
\newcommand{\VerbBar}{|}
\newcommand{\VERB}{\Verb[commandchars=\\\{\}]}
\DefineVerbatimEnvironment{Highlighting}{Verbatim}{commandchars=\\\{\}}
% Add ',fontsize=\small' for more characters per line
\usepackage{framed}
\definecolor{shadecolor}{RGB}{248,248,248}
\newenvironment{Shaded}{\begin{snugshade}}{\end{snugshade}}
\newcommand{\AlertTok}[1]{\textcolor[rgb]{0.94,0.16,0.16}{#1}}
\newcommand{\AnnotationTok}[1]{\textcolor[rgb]{0.56,0.35,0.01}{\textbf{\textit{#1}}}}
\newcommand{\AttributeTok}[1]{\textcolor[rgb]{0.77,0.63,0.00}{#1}}
\newcommand{\BaseNTok}[1]{\textcolor[rgb]{0.00,0.00,0.81}{#1}}
\newcommand{\BuiltInTok}[1]{#1}
\newcommand{\CharTok}[1]{\textcolor[rgb]{0.31,0.60,0.02}{#1}}
\newcommand{\CommentTok}[1]{\textcolor[rgb]{0.56,0.35,0.01}{\textit{#1}}}
\newcommand{\CommentVarTok}[1]{\textcolor[rgb]{0.56,0.35,0.01}{\textbf{\textit{#1}}}}
\newcommand{\ConstantTok}[1]{\textcolor[rgb]{0.00,0.00,0.00}{#1}}
\newcommand{\ControlFlowTok}[1]{\textcolor[rgb]{0.13,0.29,0.53}{\textbf{#1}}}
\newcommand{\DataTypeTok}[1]{\textcolor[rgb]{0.13,0.29,0.53}{#1}}
\newcommand{\DecValTok}[1]{\textcolor[rgb]{0.00,0.00,0.81}{#1}}
\newcommand{\DocumentationTok}[1]{\textcolor[rgb]{0.56,0.35,0.01}{\textbf{\textit{#1}}}}
\newcommand{\ErrorTok}[1]{\textcolor[rgb]{0.64,0.00,0.00}{\textbf{#1}}}
\newcommand{\ExtensionTok}[1]{#1}
\newcommand{\FloatTok}[1]{\textcolor[rgb]{0.00,0.00,0.81}{#1}}
\newcommand{\FunctionTok}[1]{\textcolor[rgb]{0.00,0.00,0.00}{#1}}
\newcommand{\ImportTok}[1]{#1}
\newcommand{\InformationTok}[1]{\textcolor[rgb]{0.56,0.35,0.01}{\textbf{\textit{#1}}}}
\newcommand{\KeywordTok}[1]{\textcolor[rgb]{0.13,0.29,0.53}{\textbf{#1}}}
\newcommand{\NormalTok}[1]{#1}
\newcommand{\OperatorTok}[1]{\textcolor[rgb]{0.81,0.36,0.00}{\textbf{#1}}}
\newcommand{\OtherTok}[1]{\textcolor[rgb]{0.56,0.35,0.01}{#1}}
\newcommand{\PreprocessorTok}[1]{\textcolor[rgb]{0.56,0.35,0.01}{\textit{#1}}}
\newcommand{\RegionMarkerTok}[1]{#1}
\newcommand{\SpecialCharTok}[1]{\textcolor[rgb]{0.00,0.00,0.00}{#1}}
\newcommand{\SpecialStringTok}[1]{\textcolor[rgb]{0.31,0.60,0.02}{#1}}
\newcommand{\StringTok}[1]{\textcolor[rgb]{0.31,0.60,0.02}{#1}}
\newcommand{\VariableTok}[1]{\textcolor[rgb]{0.00,0.00,0.00}{#1}}
\newcommand{\VerbatimStringTok}[1]{\textcolor[rgb]{0.31,0.60,0.02}{#1}}
\newcommand{\WarningTok}[1]{\textcolor[rgb]{0.56,0.35,0.01}{\textbf{\textit{#1}}}}
\usepackage{graphicx}
\ifxetex
  \usepackage[setpagesize=false, % page size defined by xetex
              unicode=false, % unicode breaks when used with xetex
              xetex]{hyperref}
\else
  \usepackage[unicode=true]{hyperref}
\fi
\hypersetup{breaklinks=true,
            bookmarks=true,
            pdfauthor={},
            pdftitle={Short report},
            colorlinks=false,
            urlcolor=blue,
            linkcolor=magenta,
            pdfborder={0 0 0}}
\urlstyle{same}  % don't use monospace font for urls

\setcounter{secnumdepth}{0}
% Pandoc toggle for numbering sections (defaults to be off)
\setcounter{secnumdepth}{0}

% Pandoc citation processing

% Pandoc header



\begin{document}
\begin{frontmatter}

  \title{Short report}
    \author[]{Hsin}
  
      
  \begin{abstract}
  
  \end{abstract}
  
 \end{frontmatter}

\hypertarget{introduction}{%
\section{Introduction}\label{introduction}}

In neuroscience research, behaviour analysis plays an important role in
validating novel treatment. Depending on animal model and therapy,
researchers choose various behaviour assessments to investigate
effectiveness of medical interventions. Gait is one of the essential
behaviour tests to examine mobility alteration in lysosomal storage
disorders, such as Niemann-Pick diseases. There are automated systems,
such as Catwalk (Noldus), that are able to collect and analyze gait data
with minimized human effort. However, Catwalk system generates numerous
variables which are around 300. Some scientists filter these variables
via hypothesis-driven approach. This only investigates certain variables
that are selected by researchers' assumption. Others merely reported
significant results. Neither can provide objective judgment of the gait
result, because selected variables were not represented the animal
model. Here, we proposed a method of variable selection to avoid
subjective conclusion. In order to distinguish relevant variables, the
mixed ANOVA analysis is introduced. The method is considered individual
subjects, age and group of a animal model effects on a study. This can
construct growth curve in wild type group and disease progression in
individual variable. In other words, the mixed ANOVA model identifies
variables which are significant different between growth curve and
disease progression. Thereafter, animal model will have its unique
variable profile. The resulted profile is according to animal model,
which means the selection process can provide a set of variables
tailoring to specific disease studies. The treatment can be inspected in
these variables which are affected by a disease. Instead, researchers
utilized biased variable selection. Interpretation of behavioral study
for inexperienced researchers is a challenging task and involved largely
personal judgment. A summarized result of assessments can help scientist
to determine the effect of treatment on animal model. We proposed a
consultation group of machine learning algorithms in order to provide a
unbiased conclusion. In this project, five classic classifiers, such as
K-Nearest Neighbors(KNN), Naive Bayes, Decision Tree, Logistic
Regression, and Support Vector Machines(SVM) were demonstrated the
consultation group concept. These algorithms indicate the likelihood of
behaviours between treated group and neither wild-type or disease
groups. These results will be the reference for researchers to make a
conclusion.

\#20/09/2021 meeting In tanganil treatment study, 9 week of NPC mice,
negative control group, lacked of gait data because of poor paw
detection. This brought a challenging task to analyze the effectiveness
of one week treatment from 8 week to 9 week. A method of two stages of
analysis was proposed to deal with this dataset. First Mixed anova is to
select relevant variables and followed by t-test. With selected
variables, t-test is conducted to compare treated group and wild-type
group at 9 week. In the result, the non-significant variables indicates
that the one week treatment benefits in these behavioural aspects and
vice versa. To rephrase it, this analysis is to investigate whether the
treatment can alter disease progression towards wild-type feathers while
by-passing the negative group comparison.

\hypertarget{methods}{%
\section{Methods}\label{methods}}

\hypertarget{tanganil-treament}{%
\subsubsection{Tanganil treament}\label{tanganil-treament}}

There were three groups, such as wild-type, NPC and treated groups. The
behaviour data were collected one week apart between age of 6 week old
and 8 week. The treatment was introduced at 8 week of homozygous mice.

\hypertarget{programming-environment}{%
\subsubsection{Programming environment}\label{programming-environment}}

R version : 4.0.3 library : ANOVA analysis : tydiverse, nlme packages
Machine learning algorithms : caret, DoMc packages

\hypertarget{mixed-anova-model}{%
\subsubsection{Mixed ANOVA model}\label{mixed-anova-model}}

In order to determine the relevant variables which are affected by
disease, data with wild type and disease groups in different time points
was analyzed by mixed ANOVA model. This can indicate the difference
between growth curve and disease procession. nlme package in R was
applied in here.

\begin{Shaded}
\begin{Highlighting}[]
\KeywordTok{library}\NormalTok{(nlme)}
\KeywordTok{library}\NormalTok{(tidyverse)}
\NormalTok{gait_data=nlme}\OperatorTok{::}\KeywordTok{groupedData}\NormalTok{(}\DataTypeTok{from=}\NormalTok{Y}\OperatorTok{~}\DecValTok{1}\OperatorTok{|}\NormalTok{ID,}\DataTypeTok{data=}\NormalTok{gait_data)}
\KeywordTok{aov}\NormalTok{(}\DataTypeTok{data=}\NormalTok{gait_data,Y}\OperatorTok{~}\NormalTok{Age}\OperatorTok{*}\NormalTok{Group}\OperatorTok{+}\KeywordTok{Error}\NormalTok{(ID)) }
\CommentTok{# mixed aov : repeated measures and various time points}
\end{Highlighting}
\end{Shaded}

\begin{itemize}
\tightlist
\item
  gait\_data= the data generated from Catwalk system
\item
  Y= one of variables from gait\_data
\end{itemize}

\hypertarget{t-test-analysis}{%
\subsubsection{T-test analysis}\label{t-test-analysis}}

In the second stage, t-test is for comparing treated and wild-type
groups in resulted variables from mixed ANOVA model. The programming
code is described below :

\begin{Shaded}
\begin{Highlighting}[]
\NormalTok{f <-}\StringTok{ }\KeywordTok{var.test}\NormalTok{(}\StringTok{'wild-type data'}\NormalTok{,}\StringTok{'NPC data'}\NormalTok{,}
              \DataTypeTok{ratio =} \DecValTok{1}\NormalTok{,}\DataTypeTok{alternative =} \StringTok{'two.sided'}\NormalTok{,}
              \DataTypeTok{na.action=}\KeywordTok{na.omit}\NormalTok{())}
\CommentTok{# variance test before t-test}

\ControlFlowTok{if}\NormalTok{(}\OperatorTok{!}\KeywordTok{is.na}\NormalTok{(}\StringTok{'result_var.test'}\NormalTok{)\{}
  \KeywordTok{ifelse}\NormalTok{(}\StringTok{'result_var.test'}\OperatorTok{>=}\FloatTok{0.05}\NormalTok{,}
\NormalTok{         t <-}\StringTok{ }\KeywordTok{t.test}\NormalTok{(}\StringTok{'wild-type data'}\NormalTok{,}\StringTok{'NPC data'}\NormalTok{,}
                     \DataTypeTok{var.equal =}\NormalTok{ T,}\DataTypeTok{alternative =} \StringTok{'two.sided'}\NormalTok{,}
                     \DataTypeTok{mu=}\DecValTok{0}\NormalTok{,}\DataTypeTok{na.action=}\KeywordTok{na.omit}\NormalTok{()),}
\NormalTok{         t <-}\KeywordTok{t.test}\NormalTok{(}\StringTok{'wild-type data'}\NormalTok{,}\StringTok{'NPC data'}\NormalTok{,}
                     \DataTypeTok{var.equal =}\NormalTok{ F,}\DataTypeTok{alternative =} \StringTok{'two.sided'}\NormalTok{,}
                     \DataTypeTok{mu=}\DecValTok{0}\NormalTok{,}\DataTypeTok{na.action=}\KeywordTok{na.omit}\NormalTok{()))}
\CommentTok{# use t-test according to the result of variance test.}
\end{Highlighting}
\end{Shaded}

\hypertarget{machine-learning-algorithms}{%
\subsubsection{Machine Learning
Algorithms}\label{machine-learning-algorithms}}

The data from wild-type and disease groups were fed to machine learning
algorithms. The mixed ANOVA analysis is regarded as features selection
process. The selected variables were to train the classifiers. The
training and 5 fold cross validation were executed by CARET package in
R.

\begin{Shaded}
\begin{Highlighting}[]
\KeywordTok{library}\NormalTok{(tidyverse)}
\KeywordTok{library}\NormalTok{(caret) }
\KeywordTok{library}\NormalTok{(doMC) }

\NormalTok{fitControl <-}\StringTok{ }\KeywordTok{trainControl}\NormalTok{(}
  \DataTypeTok{method =} \StringTok{'repeatedcv'}\NormalTok{,}\DataTypeTok{number =} \DecValTok{5}\NormalTok{,}
  \DataTypeTok{savePredictions =} \StringTok{'final'}\NormalTok{,        }
  \CommentTok{# save prediction for optimal tuning parameter     }
  \DataTypeTok{classProbs=}\NormalTok{T,                     }
  \CommentTok{# class probabilities}
  \DataTypeTok{summaryFunction=}\NormalTok{twoClassSummary,  }
  \CommentTok{# results binary classes summary function}
  \DataTypeTok{allowParallel =}\NormalTok{ T                 }
  \CommentTok{# multiple cores paralleling calculation}
\NormalTok{                           )}
\NormalTok{model_selected <-}\StringTok{ }\KeywordTok{c}\NormalTok{(}\StringTok{'regLogistic'}\NormalTok{,}\StringTok{'naive_bayes'}\NormalTok{,}
                    \StringTok{'glm'}\NormalTok{,}\StringTok{'xgbTree'}\NormalTok{,}\StringTok{'knn'}\NormalTok{)}
  \CommentTok{# five classifiers selected}
\NormalTok{ml_list <-}\StringTok{ }\KeywordTok{list}\NormalTok{()}
  \CommentTok{# a list to store model result}
\ControlFlowTok{for}\NormalTok{ (i }\ControlFlowTok{in} \DecValTok{1}\OperatorTok{:}\KeywordTok{length}\NormalTok{(model_selected)) \{}
\NormalTok{  ml_list[[i]] <-}\StringTok{ }
\StringTok{    }\NormalTok{caret}\OperatorTok{::}\KeywordTok{train}\NormalTok{(Group}\OperatorTok{~}\NormalTok{.,}
    \DataTypeTok{data=}\NormalTok{trainData,}\DataTypeTok{method=}\NormalTok{model_selected[[i]],}
    \DataTypeTok{trControl=}\NormalTok{fitControl,}\DataTypeTok{tuneLength=}\DecValTok{5}\NormalTok{,}\DataTypeTok{metric=}\StringTok{'ROC'}\NormalTok{)}
\end{Highlighting}
\end{Shaded}

\begin{itemize}
\tightlist
\item
  fitControl = the setting of 5 folds cross validation
\item
  model\_selected = the classifier chosen for this study
\item
  ml\_list = the result of the trained classifiers
\end{itemize}

\hypertarget{results}{%
\section{Results}\label{results}}

\begin{figure}
\centering
\includegraphics{report_Elsevier_files/figure-latex/mixed_aov-1.pdf}
\caption{The heapmap of Mixed AOV result}
\end{figure}

\begin{figure}
\centering
\includegraphics{report_Elsevier_files/figure-latex/t-test,-1.pdf}
\caption{The result of t-test : red colour indicates the significant
different between wild-type and treated group at ago of 9 week. blue
colour shows opposite result. In here, the therapy benefits in blue
variables}
\end{figure}

The two-stage method was demonstrated by tanganil one week treatment
study. This was proposed to solve the problems of variable selection and
absence of NPC data at age of 9 week. Wild-type and disease groups
firstly was analyzed by mixed anova model. The relevant variables from
Catwalk system were selected, Figure 1. These were represented the
disease affected variables. Afterward, t-test between wild-type and
treated groups at 9 week old was shown the benefits of tanganil in the
NPC animal model, Figure 2.

The unique variables profile was provided by mixed ANOVA, which mean the
profile depends on the disease model. It was shown in heatmap. The
colour codes are according to p-values of mixed ANOVA analysis.
Classifiers which were trained by resulted data from mixed ANOVA model
were K-Nearest Neighbors(KNN), Naive Bayes, Decision Tree, Logistic
Regression, and Support Vector Machines(SVM). The


\end{document}

